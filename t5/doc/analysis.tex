\newpage
\section{Theoretical Analysis}
\label{sec:analysis}

%inicio analise t3
%As already mentioned, to perform the theoretical analysis of the circuit {\bf Octave} was used. With this tool, we simulated the sine wave of the circuit after going through the transformer. Then that wave goes through the full-wave rectifier. As already mentioned, this rectifier (using the ideal diode model) transforms the signal to its absolute value. Next, using the capacitor, the signal starts to really look like a DC voltage. In figure \ref{fig:acdcoc}, it is possible to observe the output voltage of the Envelope Detector. After an initial time, the voltage oscillates near 15V.

%inicio analise t4
%To study the circuit in figure \ref{circuit}, it was first performed an operating point analysis, OP, as it was explained in classes. In this DC analysis (the source is like a short-circuit), the capacitors have infinite impedance, so they behave like an open circuit. The fact that capacitor $C_O$ behaves like an open circuit implies that the DC final output voltage is equal to 0 V, like the DC initial voltage. Bearing this in mind, the capacitors and $R_{in}$ can be removed from the OP analysis. Using a Thévenin's equivalent for the bias circuit makes the OP analysis quite simple. Considering first the common emitter, with the mesh analysis it is possible to observe that:

First, the real circuit was analysed. As already mentioned, to perform the theoretical analysis {\bf Octave} was used. This analysis was conducted considering a sine wave of amplitude 1 and phase 0 as the input. In order to compute the gain, the input impedance and the output impedance, the circuit was analysed using phasors.

It is possible to observe that the OP-AMP is inserted in a non-inverting amplifier with resistive feedback loop. Considering the ideal model for the OP-AMP, the voltage in node 3 must be equal to voltage in node 2. Also, voltage in node 2 must be:

\begin{equation}
        \widetilde{V_2} = \frac{R_4}{R_4+R_5}\widetilde{V_6} \iff \widetilde{V_6} = (1+\frac{R_5}{R_4})\widetilde{V_2}
\end{equation}

An ideal OP-AMP has infinite input impedance, so no current goes in the entries of the OP-AMP. Considering this, we can perform nodal analysis and, for node 3, we obtain: 

\begin{equation}
        (\widetilde{V_3}-\widetilde{V_i})jwC_1 + \frac{\widetilde{V_3}}{R_1} = 0 \iff \widetilde{V_3} = \frac{jwC_1R_1\widetilde{V_i}}{jwC_1R_1+1}
\end{equation}

For the node of the output voltage, we get:

\begin{equation}
        \frac{\widetilde{V_o}-\widetilde{V_6}}{R_2} + \widetilde{V_o}jwC_2 = 0 \iff \widetilde{V_o} = \frac{\widetilde{V_6}}{1+jwC_2R_2}
\end{equation}

The transfer function is then:

\begin{equation}
        \frac{\widetilde{V_o}}{\widetilde{V_i}} = (1+\frac{R_5}{R_4})\frac{jwC_1R_1}{jwC_1R_1+1}\frac{1}{1+jwC_2R_2}
\end{equation}

%With this, it's possible to write the following matrix:

%\begin{equation}
 % \begin{bmatrix}
  %  1 & -1 & 0 & 0 \\
   % 1 & 0 & -\frac{R_4}{R_4+R_5} & 0 \\
    %0 & jwC_1+\frac{1}{R_1} & 0 & 0 \\
    %0 & 0 & -\frac{1}{R_2} & jwC_2+\frac{1}{R_2}  
  %\end{bmatrix}
  %\begin{bmatrix}
%    \widetilde{V_2} \\
 %   \widetilde{V_3} \\
  %  \widetilde{V_6} \\
   % \widetilde{V_o}
%  \end{bmatrix}
 % =
  %\begin{bmatrix}
   % 0 \\
    %0 \\
%    jwC_1\widetilde{V_i} \\
 %   0
  %\end{bmatrix}
%  \label{pha}
%\end{equation}

%By solving the matrix it is then possible to compute the gain for every frequency.

The central frequency will then be calculated as:

\begin{equation}
        w_c = \sqrt{w_Lw_H}
\end{equation}

where $w_L=\frac{1}{R_1C_1}$ and $w_H=\frac{1}{R_2C_2}$.

The input impedance will be: 

\begin{equation}
        z_{in} = \frac{\widetilde{V_i}}{\widetilde{I_i}} = \frac{\widetilde{V_i}}{(\widetilde{V_i}-\widetilde{V_3})jwC_1}
\end{equation}

The output impedance is calculated considering that the input source is off, is like a short-circuit. That results in no current in $C_1$, $R_1$ and $R_4$. $R_5$ is in parallel with the OP-AMP. Ideally, the OP-AMP has 0 output impedance, so the parallel with $R_5$ results in 0 impedance. So, in the end, what is left is the parallel of $R_2$ and $C_2$. Consequently, the output impedance is:

\begin{equation}
        z_{out} = \frac{1}{\frac{1}{R_2}+jwC_2}
\end{equation}

With this analysis, the following results were obtained (gain and impedances were calculated for 1000 Hz):

\begin{table}[H]
  \centering
  \begin{tabular}{|c|c|}
    \hline
        {\bf Name} & {\bf Value} \\
        \hline
        \hline
        \input{centralf}
        \hline
  \end{tabular}
  \caption{Results}
  \label{teo_results}
\end{table}

The frequency response was also calculated and the following was obtained:

\begin{figure}[H]
        \centering
        \includegraphics[width=0.6\textwidth]{Vo_Vi.eps}
        \caption{Frequency response - Gain}
        \label{tfrg}
\end{figure}

\begin{figure}[H]
        \centering
        \includegraphics[width=0.6\textwidth]{phase.eps}
        \caption{Frequency response - Phase}
        \label{tfrf}
\end{figure}


For the optimized circuit, the following results were calculated:

\begin{table}[H]
  \centering
  \begin{tabular}{|c|c|}
    \hline
        {\bf Name} & {\bf Value} \\
        \hline
        \hline
        \input{centralf_opt}
        \hline
  \end{tabular}
  \caption{Results}
  \label{teo_results}
\end{table}

\begin{figure}[H]
        \centering
        \includegraphics[width=0.6\textwidth]{Vo_Vi_opt.eps}
        \caption{Frequency response - Gain}
        \label{tfrg}
\end{figure}

\begin{figure}[H]
        \centering
        \includegraphics[width=0.6\textwidth]{phase_opt.eps}
        \caption{Frequency response - Phase}
        \label{tfrf}
\end{figure}
