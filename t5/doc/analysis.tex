\newpage
\section{Theoretical Analysis}
\label{sec:analysis}

%inicio analise t3
%As already mentioned, to perform the theoretical analysis of the circuit {\bf Octave} was used. With this tool, we simulated the sine wave of the circuit after going through the transformer. Then that wave goes through the full-wave rectifier. As already mentioned, this rectifier (using the ideal diode model) transforms the signal to its absolute value. Next, using the capacitor, the signal starts to really look like a DC voltage. In figure \ref{fig:acdcoc}, it is possible to observe the output voltage of the Envelope Detector. After an initial time, the voltage oscillates near 15V.

%inicio analise t4
%To study the circuit in figure \ref{circuit}, it was first performed an operating point analysis, OP, as it was explained in classes. In this DC analysis (the source is like a short-circuit), the capacitors have infinite impedance, so they behave like an open circuit. The fact that capacitor $C_O$ behaves like an open circuit implies that the DC final output voltage is equal to 0 V, like the DC initial voltage. Bearing this in mind, the capacitors and $R_{in}$ can be removed from the OP analysis. Using a Thévenin's equivalent for the bias circuit makes the OP analysis quite simple. Considering first the common emitter, with the mesh analysis it is possible to observe that:

