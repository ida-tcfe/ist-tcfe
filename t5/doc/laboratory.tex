\newpage

\section{Real circuit}
\label{sec:laboratory}

For the purpose of trying out this circuit, we chose the following parameters:

\begin{table}[H]
\centering
\begin{tabular}{|c|c|}
        \hline
        Parameter & Value \\
        \hline
        $C_1 (nF)$ & 220 \\
        $R_1 (k\Omega)$ & 1 \\
        $C_2 (nF)$ & 220 \\
        $R_2 (k\Omega)$ & 0.5 \\
        $R_4 (k\Omega)$ & 1 \\
        $R_5 (k\Omega)$ & 100 \\
        \hline
\end{tabular}
\caption{Parameters for the simple circuit}
\label{simple_par}
\end{table}

By varying the frequency using a signal generator and measuring the peak to peak amplitudes of $v_i$ and $v_o$, the gain as a function of the frequency could be determined. Below is a table which includes the data we retrieved ($f$, $2|v_1|$ and $2|v_2|$) and the calculated gain, in $dB$.

\begin{table}[H]
\centering
\begin{tabular}{|c|c|c|c|}
        \hline
        $f\;(Hz)$ & $2|v_1|\;(V)$ & $2|v_2|\;(V)$ & $(\frac{v_2}{v_1})_{dB}$ \\
        \hline
        500 & 0.105 & 5.5 & 34.28 \\
        600 & 0.105 & 5.9 & 34.99 \\
        700 & 0.105 & 6.2 & 35.42 \\
        800 & 0.105 & 6.2 & 35.42 \\
        900 & 0.105 & 6.2 & 35.42 \\
        1000 & 0.105 & 6.2 & 35.42 \\
        1100 & 0.105 & 6 & 35.14 \\
        1200 & 0.104 & 5.9 & 35.08 \\
        1300 & 0.104 & 5.7 & 34.78 \\
        1400 & 0.104 & 5.55 & 34.55 \\
        1500 & 0.104 & 5.4 & 34.31 \\
        2000 & 0.104 & 4.6 & 32.91 \\
        2500 & 0.104 & 3.9 & 31.48 \\
        3000 & 0.104 & 3.35 & 30.16 \\
        3500 & 0.104 & 2.9 & 28.91 \\
        4000 & 0.104 & 2.6 & 27.96 \\
        4500 & 0.104 & 2.3 & 26.89 \\
        5000 & 0.104 & 2 & 25.68 \\
        5500 & 0.104 & 1.8 & 24.76 \\
        6000 & 0.104 & 1.6 & 23.74 \\
        10000 & 0.103 & 1 & 19.74 \\
        \hline
\end{tabular}
\caption{Data collected}
\label{data}
\end{table}

\begin{figure}[H]
\centering
\includegraphics[width=0.6\textwidth]{gain}
\caption{Data collected}
\label{data_graph}
\end{figure}
