\section{Conclusion}
\label{sec:conclusion}

%parte do t4
%This laboratory assignment had as the main objective the construction and analysis of an audio amplifier. The chosen design is present in figure \ref{circuit}. The goal was to maximise the voltage gain and the bandwidth at which the circuit operates at ``full capacity'', while minimising the cost and the lower cutoff frequency, preferably to less than $20\;Hz$, the lower limit of the audible range.
%The merit figure of our first circuit is: \input{../sim/merit}. For our second circuit, it's \input{../sim/merit_mb}.
%Some deviations between the simulation and theoretical results were expected, due to the different model used by {\bf Ngspice} for transistors. In the theoretical analysis, we used very simple models. These deviations can be observed not only in table \ref{error1_res}, but also by comparing figures \ref{vg} and \ref{fig:frequencyresp} and the gains and impedances computed.
%All in all, the objectives were met. With more time and a lengthier study, better results could be obtained, and other solutions implemented.

\begin{figure}[H]
        \centering
        \includegraphics[width=0.8\textwidth]{gain_all.eps}
        \caption{Data comparison}
        \label{alldata}
\end{figure}

\begin{table}[H]
  \centering
  \begin{tabular}{|c|c|c|c|c|}
    \hline
        $V$ & $V_t$ & $V_s$ & $|V_t-V_s|$ & $Error (\%)$ \\
        \hline
        \hline
        $V_{B1}$ (V) & 0.81181 & 1.529471 & 0.717661 & 46.92 \\ 
 \hline 
$V_{C1}$ (V) & 10.776937 & 11.19816 & 0.42122299999999946 & 3.76 \\ 
 \hline 
$V_{E1}$ (V) & 0.11181 & 0.8257281 & 0.7139181 & 86.46 \\ 
 \hline 
$V_{E2}$ (V) & 11.476937 & 11.96096 & 0.48402300000000054 & 4.05 \\ 
 \hline 
$I_{B1}$ (V) & 0.006222 & 4.350727e-05 & 0.00617849273 & 14201.06 \\ 
 \hline 
$I_{R1}$ (V) & 0.00048 & 0.0001163392 & 0.0003636608 & 312.59 \\ 
 \hline 
$I_{R2}$ (V) & 0.005742 & 7.283194e-05 & 0.005669168059999999 & 7783.90 \\ 
 \hline 
$I_{C1}$ (V) & 1.111876 & -0.00821377 & 1.12008977 & 13636.73 \\ 
 \hline 
$I_{E1}$ (V) & 1.118098 & 0.008257281 & 1.1098407190000001 & 13440.75 \\ 
 \hline 
$I_{E2}$ (V) & 0.174354 & -0.0390369 & 0.2133909 & 546.64 \\ 
 \hline 

        \hline
  \end{tabular}
  \caption{Theoretical values ($V_t$) and Simulation values ($V_s$) of the operating point analysis (table produced with {\bf Python})  - The absolute deviation and error presented here are rounded up to one significant digit, for ease of interpretation.}
  \label{error1_res}
\end{table}
