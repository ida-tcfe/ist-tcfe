\newpage

\section{Introduction}
\label{sec:introduction}

The main objective of this laboratory assignment is to design a bandpass filter, using an OP-AMP. This filter should be centered at $1\;kHz$ and produce a gain of $40\;dB$ at this frequency.

The secondary goal is to achieve the best performance at the lowest cost. This is quantified by maximizing the following function:

\begin{equation}
  M = \frac{1}{cost \cdot gainDeviation \cdot centralFrequencyDeviation}
\end{equation}

\begin{gather*}
  cost = cost\;of\;resistors + cost\;of\;capacitors + cost\;of\;transistors \\
  cost\;of\;resistors = 1 MU/k\Omega \\
  cost\;of\;capacitors = 1 MU/\mu F \\
  cost\;of\;transistors = 0.1 MU/transistor
\end{gather*}

In Section~\ref{sec:laboratory} a basic filter circuit is introduced and the study performed in the laboratory, using this circuit, is presented. This is followed by Section~\ref{sec:simulation}, where the same circuit is simulated. Furthermore, in an effort to improve our results, an optimized circuit is obtained and presented. A theoretical analysis is then performed, considering either circuit, in Section~\ref{sec:analysis}. The results obtained for the optimized circuit via simulation and theoretical analysis are compared in Section~\ref{sec:conclusion}. Here, $M$ is presented, allowing for some considerations to be made about the studies performed. The simulation was run on {\bf Ngspice} and the theoretical calculations were performed with {\bf Octave}.

%The main objective of this laboratory assignment is to design an audio amplifier. This circuit should be able to amplify $10\;mV$ of alternating current at a wide range of frequencies. To this purpose, three types of components will be employed: resistors, capacitors and transistors.
%The secondary goal is to achieve the best performance at the lowest cost. This is quantified by maximizing the following function:
%\begin{equation}
%  M = \frac{voltageGain \cdot bandwidth}{cost \cdot lowerCutoffFrequency}
%\end{equation}
%\begin{gather*}
%  cost = cost\;of\;resistors + cost\;of\;capacitors + cost\;of\;transistors \\
%  cost\;of\;resistors = 1 MU/k\Omega \\
%  cost\;of\;capacitors = 1 MU/\mu F \\
%  cost\;of\;transistors = 0.1 MU/transistor
%\end{gather*}
%In Section~\ref{sec:simulation}, the amplifier is presented and its behaviour is studied using a simulation. This is followed by a theoretical analysis of the circuit, in Section~\ref{sec:analysis}. Both are compared in the final section, Section ~\ref{sec:conclusion}. Here, $M$ is also presented, allowing for some considerations to be made about the circuit designed. The simulation was run by using {\bf Ngspice} and the theoretical calculations were performed with {\bf Octave}.
