\newpage

\section{Simulation Analysis}
\label{sec:simulation}

\subsection{Operating Point Analysis}
\label{sec:op_point}

\subsubsection{($t < 0$)}

After the theoretical analysis, we proceeded to simulate the circuit using {\bf Ngspice} in order to compare the results with the theoretical predictions. The results for $t < 0$ are shown in Table~\ref{tab:op1}.

\begin{table}[H]
	\centering
	\begin{tabular}{|l|c|}
		\hline    
		    {\bf Name} & {\bf Value} \\
                    \hline
                    \hline
		\input{../sim/op_ng1_tab}
	\end{tabular}
	\caption{Operating point. The '$n$' variables represent the voltages determined at the nodes. The '$r$' variables correspond to the current flowing through each resistor.}
	\label{tab:op1}
\end{table}

With this, we are able to deduce the value of the potencial on nodes 6 and 8, i.e. to know the tension of the fully charged capacitor, which, according to this simulation, is expected to be $8.624089\;V$.

\subsubsection{($t = 0$)}

In this part, the simulation was conducted with $v_s(0) = 0$ and by substituting the capacitor with an independent voltage source with a tension about the one calculated above for the fully charged capacitor. With this we are not only able to calculate the voltage at the nodes at ($t = 0$), which would otherwise not be possible since the node voltages can be descontinuous (with the expection of the capacitor nodes), but also able to deduce the equivalente resistance ($R_{eq}$) by which the capacitor discharges. This enables us to later simulate the natutal solution when doing \emph{transient analysis}.

\begin{table}[H]
	\centering
	\begin{tabular}{|l|c|}
		\hline    
		    {\bf Name} & {\bf Value} \\
                    \hline
                    \hline
		\input{../sim/op_ng2_tab}
	\end{tabular}
	\caption{Operating point. The '$n$' variables represent the voltages determined at the nodes. The '$r$' variables correspond to the current flowing through each resistor.}
	\label{tab:op2}
\end{table}

\begin{figure}[H]
  \centering
  \includegraphics[width=0.6\linewidth]{trans3.eps}
  \caption{Natural response}
  \label{fig:nat_sim}
\end{figure}

\subsubsection{($ t \geq 0$)}

\begin{figure}[H]
  \centering
  \includegraphics[width=0.6\linewidth]{trans4.eps}
  \caption{Total response and source signal (for comparison)}
  \label{fig:tot:sim}
\end{figure}

For the forced solution, the magnitudes and phase delays obtained were as follows (for ease of comparison, the table features $\phi = \frac{\pi}{2} - \phi'$):

\begin{table}[H]
	\centering
	\begin{tabular}{|l|c|}
		\hline    
		    {\bf Name} & {\bf Value} \\
                    \hline
                    \hline
		\input{../sim/op_ng4_tab}
	\end{tabular}
	\caption{AC analysis ([Phi]=rad)}
	\label{tab:op4}
\end{table}

\subsection{Frequency response}

\begin{figure}[H]
  \centering
  \includegraphics[width=0.6\textwidth]{ac_mag5.eps}
  \caption{Magnitude frequency response}
  \label{freq_resp_mag_sim}
\end{figure}

\begin{figure}[H]
  \centering
  \includegraphics[width=0.6\textwidth]{ac_phase5.eps}
  \caption{Phase frequency response}
  \label{freq_resp_pha_sim}
\end{figure}

We can see that the graphs \ref{freq_resp_mag_sim} and \ref{freq_resp_pha_sim} follow the equations refered in the analysis section. It's important to note that magnitude is in \emph{dB} (so as the gain goes to 0 that correspondes to the values in the graph going to -$\infty$).

As expected we see that in both the simulation and theoretical analysis the circuit behaves as a low pass filter, when evaluated at $v_6$ or at the capacitor's terminals ($v_c$).	
