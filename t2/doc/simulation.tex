\newpage

\section{Simulation Analysis}
\label{sec:simulation}

\subsection{Operating Point Analysis}
\label{sec:op_point}

\subsubsection{($t < 0$)}

After the theoretical analysis, we proceeded to simulate the circuit using {\bf Ngspice} in order to compare the results with the theoretical predictions. The results for $t < 0$ are shown in Table~\ref{tab:op}.

\begin{table}[H]
	\centering
	\begin{tabular}{|l|c|}
		\hline    
		    {\bf Name} & {\bf Value} \\
                    \hline
                    \hline
		\input{../sim/op_ng1_tab}
	\end{tabular}
	\caption{Operating point. The '$n$' variables represent the voltages determined at the nodes. The '$r$' variables correspond to the current flowing through each resistor.}
	\label{tab:op1}
\end{table}

\subsubsection{($t = 0$)}

In this part, the simulation was conducted with $v_s(0) = 0$. 

\begin{table}[H]
	\centering
	\begin{tabular}{|l|c|}
		\hline    
		    {\bf Name} & {\bf Value} \\
                    \hline
                    \hline
		\input{../sim/op_ng2_tab}
	\end{tabular}
	\caption{Operating point. The '$n$' variables represent the voltages determined at the nodes. The '$r$' variables correspond to the current flowing through each resistor.}
	\label{tab:op2}
\end{table}

\begin{figure}[H]
  \centering
  \includegraphics[width=0.8\linewidth]{trans3.eps}
  \caption{Natural response}
  \label{fig:nat}
\end{figure}

\subsubsection{($ t \geq 0$)}

\begin{figure}[H]
  \centering
  \includegraphics[width=0.8\linewidth]{trans4.eps}
  \caption{Total response and source signal (for comparison)}
  \label{fig:tot}
\end{figure}
