\section{Theoretical Analysis}
\label{sec:analysis}

\subsection{Node analysis method}
\label{sec:node}

\begin{table}[H]
  \centering
  \begin{tabular}{|c|c|}
    \hline
        {\bf Name} & {\bf Value} \\
        \hline
        \hline
        $V_2\;(V)$ & $5.070727$ \\ 
\hline
$V_3\;(V)$ & $4.825468$ \\ 
\hline
$V_4\;(V)$ & $4.304579$ \\ 
\hline
$V_5\;(V)$ & $4.860091$ \\ 
\hline
$V_6\;(V)$ & $8.720760$ \\ 
\hline
$V_7\;(V)$ & $-2.939898$ \\ 
\hline
$V_8\;(V)$ & $-1.950198$ \\ 
\hline
$V_b\;(V)$ & $-0.034623$ \\ 
\hline
$I_b\;(V)$ & $-0.000251$ \\ 
\hline
$I_c\;(V)$ & $0.000969$ \\ 
\hline
$V_c\;(V)$ & $7.799989$ \\ 

        \hline
  \end{tabular}
  \caption{Theoretical results}
  \label{mesh_res}
\end{table}

\subsection{Capacitor behaviour}
\label{sec:Req}

In order to ascertain how the capacitor behaves, the equivalent resistor seen by it was determined. To do this, a tension source was added in lieu of the capacitor, and the response current of the remainder of the circuit was obtained, via node analysis methods.

For this procedure it was considered that $t=0\;s$, at which time the capacitor is fully charged, and so the voltage difference at it's terminals is known from \ref{sec:node} and constant (static analysis). At this time, it is also known that $V_s=0\;V$ - it is the exact time at which the source shifts from constant to sinusoidal.

% Boneco

By applying KCL to node 6, $I_x$ can be calculated:

\begin{equation}
  I_x = \frac{V_5-V_6}{R_5}-I_b = \frac{V_5-V_6}{R_5} - K_b \cdot (V_2-V_5)
\end{equation}

$R_{eq}$ is then obtained from Ohm's Law:

\begin{equation}
  R_{eq} = \frac{V_x}{I_x}
\end{equation}

The results obtained were as follows:

\begin{table}[H]
  \centering
  \begin{tabular}{|c|c|}
    \hline
        {\bf Name} & {\bf Value} \\
        \hline
        \hline
        \input{../mat/nodeReq}
        \hline
  \end{tabular}
  \caption{Theoretical results}
\end{table}

\begin{table}[H]
  \centering
  \begin{tabular}{|c|c|}
    \hline
        {\bf Name} & {\bf Value} \\
        \hline
        \hline
        \input{../mat/nodeReq2}
        \hline
  \end{tabular}
  \caption{Theoretical results}
\end{table}

As expected, the current value $I_x$ is negative, thus making $P_x$ negative: the voltage source supplies energy to the circuit.

