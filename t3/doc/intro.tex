\section{Introduction}
\label{sec:introduction}

% state the learning objective

The main objetive of this laboratory assignment is to design an ac/dc converter. This circuit should be able to convert 230 V of alternating current at a frequency of 50 Hz (standard domestic eletricity) to 12 V of direct current. To this purpose, three types of components will be employed: resistors, capacitors and diodes. The secondary goal is to achieve the best perfomance at the lowest cost. This is quantified by minimizing the following function:

\begin{equation}
  M = \frac{1}{cost \cdot (ripple(v_o) + average(v_o-12) + 10^{-6})}
\end{equation}

\begin{gather*}
  cost = cost\;of\;resistors + cost\;of\;capacitors + cost\;of\;diodes \\
  cost\;of\;resistors = 1 MU/k\Omega \\
  cost\;of\;capacitors = 1 MU/\mu F \\
  cost\;of\;diodes = 0.1 MU/diode
\end{gather*}

In Section~\ref{sec:simulation}, the converter and it's behaviour are presented. This is followed by a theoretical analysis of the circuit, in Section~\ref{sec:analysis}. Both are compared in the final section, Section ~\ref{sec:conclusion}. Here, $M$ is also presented, allowing for some considerations to be made about the circuit designed. The simulation was run by using {\bf Ngspice} and the theoretical calculations were performed with {\bf Octave}.
