\newpage

\section{Simulation Analysis}
\label{sec:simulation}

The basic template for an amplifier is as follows:

\begin{figure}[H]
        \centering
        \includegraphics[width=\textwidth, trim = 0 4cm 0 4cm, clip]{template.pdf}
        \caption{Template}
        \label{template}
\end{figure}

\subsection{Common emitter stage}

The purpose of the common emitter amplifier is to amplify the signal with linear DC gain. To this purpose the following circuit can be employed:

\begin{figure}[H]
        \centering
        \includegraphics[width=\textwidth, trim = 0cm 2cm 0cm 2cm, clip]{ces.pdf}
        \caption{Common emitter stage}
        \label{ces}
\end{figure}

\subsection{Common collector stage}

The purpose of the common collector amplifier is to supply enough current to the load. To this purpose the following circuit can be employed:

\begin{figure}[H]
        \centering
        \includegraphics[width=\textwidth, trim = 0cm 4cm 0cm 0cm, clip]{ccs.pdf}
        \caption{Common collector stage}
        \label{ccs}
\end{figure}

\subsection{Final result}

Combining these two circuits, a very good amplifier can be designed:

\begin{figure}[H]
        \centering
        \includegraphics[width=\textwidth, trim = 0cm 0cm 0cm 0cm, clip]{circuit.pdf}
        \caption{Amplifier}
        \label{circuit}
\end{figure}

By a lengthy process of trial and error, attempting to maximize the merit figure, while also keeping the input impedance high and the output impedance low, gave us the following parameters:

\begin{table}[H]
        \centering
        \begin{tabular}{|c|c|}
        \hline
        Parameter & Value \\
        \hline
        \input{../sim/parameters}
        \hline
        \end{tabular}
        \caption{Parameters}
        \label{param}
\end{table}

\begin{table}[H]
  \centering
  \begin{tabular}{|c|c|}
    \hline
        \input{../sim/impedances}
        \hline
  \end{tabular}
  \caption{Impedances}
  \label{tab:sim_imp}
\end{table}

\begin{table}[H]
  \centering
  \begin{tabular}{|c|c|}
    \hline
        {\bf Name} & {\bf Value} \\
        \hline
        \hline
        n2 (V) & 5.070727e+00\\ \hline
n3 (V) & 4.825468e+00\\ \hline
n4 (V) & 4.304579e+00\\ \hline
n5 (V) & 4.860091e+00\\ \hline
n6 (V) & 8.720760e+00\\ \hline
n7 (V) & -2.93990e+00\\ \hline
n8 (V) & -1.95020e+00\\ \hline
Vc (V) & 7.799989e+00\\ \hline
r1 (A) & -2.40330e-04\\ \hline
r2 (A) & -2.51475e-04\\ \hline
r3 (A) & -1.11453e-05\\ \hline
r4 (A) & -1.20919e-03\\ \hline
r5 (A) & -1.25166e-03\\ \hline
r6 (A) & 9.688613e-04\\ \hline
r7 (A) & 9.688613e-04\\ \hline
Ib (A) & -2.51475e-04\\ \hline

        \hline
  \end{tabular}
  \caption{Simulation results}
\end{table}

\begin{figure}[H]
\centering
\includegraphics[width=0.8\textwidth]{vo2f.eps}
\caption{Voltage gain}
\label{vg}
\end{figure}

\begin{table}[H]
  \centering
  \begin{tabular}{|c|c|}
    \hline
        \input{../sim/out}
        \hline
  \end{tabular}
  \caption{Voltage gain, lower cutoff frequency and bandwidth}
  \label{tab:res_sim}
\end{table}

Afterwards, we also designed a second circuit, with only merit maximisation in mind:

\begin{table}[H]
        \centering
        \begin{tabular}{|c|c|}
        \hline
        Parameter & Value \\
        \hline
        \input{../sim/parameters_mb}
        \hline
        \end{tabular}
        \caption{Parameters}
        \label{param_mb}
\end{table}

\begin{table}[H]
  \centering
  \begin{tabular}{|c|c|}
    \hline
        \input{../sim/impedances_mb}
        \hline
  \end{tabular}
  \caption{Impedances}
  \label{tab:sim_imp_mb}
\end{table}

\begin{figure}[H]
\centering
\includegraphics[width=0.8\textwidth]{vo2f_mb.eps}
\caption{Voltage gain}
\label{vg_mb}
\end{figure}

\begin{table}[H]
  \centering
  \begin{tabular}{|c|c|}
    \hline
        \input{../sim/out_mb}
        \hline
  \end{tabular}
  \caption{Voltage gain, lower cutoff frequency and bandwidth}
  \label{tab:res_sim_mb}
\end{table}

As one can see, with a bigger merit also resulted in a lower cutoff frequency and a wider bandwith. The gain became slighly smaller. As the goal would be for te lower cutoff frequency to be below $20Hz$ there is till some room for improvement. The input resistor (real part of the impedance) is also bigger, but unfortunately so is the output one, which would result in more dissipated power.
