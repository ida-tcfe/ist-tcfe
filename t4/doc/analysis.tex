\newpage
\section{Theoretical Analysis}
\label{sec:analysis}

%inicio analise t3
%As already mentioned, to perform the theoretical analysis of the circuit {\bf Octave} was used. With this tool, we simulated the sine wave of the circuit after going through the transformer. Then that wave goes through the full-wave rectifier. As already mentioned, this rectifier (using the ideal diode model) transforms the signal to its absolute value. Next, using the capacitor, the signal starts to really look like a DC voltage. In figure \ref{fig:acdcoc}, it is possible to observe the output voltage of the Envelope Detector. After an initial time, the voltage oscillates near 15V.

To study the circuit in figure \ref{circuit}, it was first performed an operating point analysis, OP, as it was explained in classes. In this DC analysis (the source is like a short-circuit), the capacitors have infinite impedance, so they behave like an open circuit. The fact that capacitor $C_O$ behaves like an open circuit implies that the DC final output voltage is equal do 0 V, like the DC initial voltage. Bearing this in mind, the capacitors and $R_{in}$ can be removed from the OP analysis. Using a Thévenin's equivalent for the bias circuit makes the OP analysis quite simple. Considering first the common emitter, with the mesh analysis it is possible to observe that:

\begin{equation}
        -V_{eq}+R_{eq}I_B+V_{BEON}+R_EI_E = 0
\end{equation}

In the equation above, $I_B$ is the base current and $I_E$ is the emitter current of the considered transistor. $V_{BEON}$ was estimated to be $0.7\;V$. By using the equation $I_E=(1+\beta_F)I_B ~\refstepcounter{equation}(\theequation) \label{fluxomagnetico}$, the following is obtained:

\begin{equation}
        I_B = \frac{V_{eq}-V_{BEON}}{R_{eq}+(1+\beta_F)R_E}
\end{equation}

With other transistor equation that gives the collector current, $I_C=\beta_FI_B ~\refstepcounter{equation}(\theequation) \label{fluxomagnetico}$, it is possible to compute all DC voltages and currents in the common emitter circuit. The output voltage of this stage will simply be $V_{O1} = V_{CC}-R_CI_C ~\refstepcounter{equation}(\theequation) \label{fluxomagnetico}$.

The common collector stage is simple to analyse. The mesh analysis gives that:

\begin{equation}
        V_{O1}-V_{cc}+R_{out}I_E+V_{EBON} = 0 \iff I_E = \frac{V_{CC}-V_{EBON}-V_{O1}}{R_E}
\end{equation}

Using the already mentioned equations for the transistor (and Ohm's Law) it is then possible to calculate all the voltages and currents of the circuit. Finally, it can be observed in table \ref{opteorico} the results obtained for the OP analysis of the circuit in figure \ref{circuit}.

\begin{table}[H]
  \centering
  \begin{tabular}{|c|c|}
    \hline
        {\bf Name} & {\bf Value} \\
        \hline
        \hline
        \input{../mat/ft4}
        \hline
  \end{tabular}
  \caption{Theoretical results OP}
  \label{opteorico}
\end{table}

This OP analysis is very important to compute the parameters necessary to perform the incremental analysis.

As derived in classes, the gain for the common emitter is equal to:

\begin{equation}
        A_v = -\frac{g_m}{\frac{1}{ro}+\frac{1}{R_C}}\frac{1}{(\frac{1}{r_{\pi}}+\frac{1}{R_{eq}})(R_{in}+\frac{1}{\frac{1}{r_{\pi}}+\frac{1}{R_{eq}}})}
\end{equation}

The common emitter input impedance is the parallel of $R_{eq}$ and $r_{\pi}$ and its output impedance is the parallel of $r_o$ and $R_C$. Using {\bf Octave}, the following results were obtained for the common emitter:

\begin{table}[H]
  \centering
  \begin{tabular}{|c|c|}
    \hline
        \input{../mat/emitter}
        \hline
  \end{tabular}
  \caption{Gain and impedances - Common emitter}
  \label{gainemitter}
\end{table}

As desired, the input impedance is bigger than $R_{in}$, however it would be better if the input impedance was bigger.

The gain for the common collector is easy to compute using the conductances:

\begin{equation}
        A_v = \frac{g_{\pi}+g_m}{g_{\pi}+g_m+g_o+g_{out}}
\end{equation}

The input and output impedances are calculated using the following equations:

\begin{equation}
        Z_I = \frac{g_{\pi}+g_m+g_o+g_{out}}{(g_o+g_{out})g_{\pi}}
\end{equation}

\begin{equation}
        Z_O = \frac{1}{g_{\pi}+g_m+g_o+g_{out}}
\end{equation}

The following results were obtained:

\begin{table}[H]
  \centering
  \begin{tabular}{|c|c|}
    \hline
        \input{../mat/collector}
        \hline
  \end{tabular}
  \caption{Gain and impedances - Common collector}
  \label{gaincollector}
\end{table}

By observing the two last tables we can conclude that the two stages can be connected without significant signal loss, because the output impedance of the common emitter is small when compared with the input impedance of the common collector. It is also worth referring that the output impedance of the common collector is, as desired, quite small.

The gain and impedances obtained for the complete circuit are presented in the table \ref{gaintotal}. The input impedance is equal to the input impedance of the common emitter, $Z_{O1}$. The gain and the output impedance are calculated with these equations, where the parameters with index 1 correspond to the common collector and the parameters with index 2 correspond to the common collector:

\begin{equation}
        A_v = \frac{\frac{1}{r_{\pi2}+Z_{O1}}+\frac{g_{m2}r_{\pi2}}{r_{\pi2}+Z_{O1}}}{\frac{1}{r_{\pi2}+Z_{O1}}+\frac{1}{R_{out}}+\frac{1}{r_{o2}}+\frac{g_{m2}r_{\pi2}}{r_{\pi2}+Z_{O1}}} A_{v1}
\end{equation}

\begin{equation}
        Z_O = \frac{1}{\frac{g_{m2}r_{\pi2}}{r_{\pi2}+Z_{O1}}+g_{o2}+g_{out}+\frac{1}{r_{\pi2}+Z_{O1}}}
\end{equation}

\begin{table}[H]
  \centering
  \begin{tabular}{|c|c|}
    \hline
        \input{../mat/total}
        \hline
  \end{tabular}
  \caption{Gain and impedances - Complete circuit}
  \label{gaintotal}
\end{table}

Incremental analysis allowed also to make a graph of the frequency response. However, with the model studied in classes for the transistor in incremental analysis, the frequency response stabilized until infinity, instead of going down eventually like it happens in the simulation.

{\bf Ngspice} uses a quite complex model for the transistor, with lots of parameters, including capacitances. To try to have a theoretical frequency response more similar to the one provided by {\bf Ngspice}, the following model was considered for the transistor, where two capacitors were added to the original model. 

\begin{figure}[H]
        \centering
        \includegraphics[width=\textwidth, trim = 0 5cm 0 5cm, clip]{transistor.pdf}
        \caption{Transistor incremental model}
        \label{transistorinc}
\end{figure}

With the model presented, the following graph was obtained for the frequency response.

\begin{figure}[H]
  \centering
  \includegraphics[width=0.8\linewidth]{Vo_Vi.eps}
  \caption{Theoretical frequency response}
  \label{fig:frequencyresp}
\end{figure}
