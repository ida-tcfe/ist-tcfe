\newpage
\section{Theoretical Analysis}
\label{sec:analysis}

%inicio analise t3
%As already mentioned, to perform the theoretical analysis of the circuit {\bf Octave} was used. With this tool, we simulated the sine wave of the circuit after going through the transformer. Then that wave goes through the full-wave rectifier. As already mentioned, this rectifier (using the ideal diode model) transforms the signal to its absolute value. Next, using the capacitor, the signal starts to really look like a DC voltage. In figure \ref{fig:acdcoc}, it is possible to observe the output voltage of the Envelope Detector. After an initial time, the voltage oscillates near 15V.

To study the circuit in figure \ref{circuit}, it was first performed an operating point analysis, OP. In this DC analysis (the source is like a short-circuit), the capacitors have infinite impedance, so they behave like an open circuit. As a consequence, the capacitors and $R_{in}$ can be removed from the OP analysis. Using a Thévenin's equivalent for the bias circuit makes the OP analysis quite simple. Considering first the common emmiter, with the mesh analysis it is possible to observe that:

\begin{equation}
        -V_{eq}+R_{eq}I_B+V_{BEON}+R_EI_E = 0
\end{equation}

In the equation above, $I_B$ is the base current and $I_E$ is the emmiter current of the considered transistor. $V_{BEON}$ was estimated to be $0.7\;V$. By using the equation $I_E=(1+\beta_F)I_B ~\refstepcounter{equation}(\theequation) \label{fluxomagnetico}$, the following is obtained:

\begin{equation}
        I_B = \frac{V_{eq}-V_{BEON}}{R_{eq}+(1+\beta_F)R_E}
\end{equation}

With other transistor equation that gives the collector current, $I_C=\beta_FI_B ~\refstepcounter{equation}(\theequation) \label{fluxomagnetico}$, it is possible to compute all DC voltages and currents in the common emmiter circuit. The output voltage of this stage will simply be $V_O = V_{CC}-R_CI_C ~\refstepcounter{equation}(\theequation) \label{fluxomagnetico}$.

\begin{table}[H]
  \centering
  \begin{tabular}{|c|c|}
    \hline
        {\bf Name} & {\bf Value} \\
        \hline
        \hline
        \input{../mat/ft4}
        \hline
  \end{tabular}
  \caption{Theoretical results OP}
  \label{opteorico}
\end{table}
