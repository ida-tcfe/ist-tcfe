\newpage

\section{Introduction}
\label{sec:introduction}

% state the learning objective

The main objective of this laboratory assignment is to design an audio amplifier. This circuit should be able to amplify $10\;mV$ of alternating current at a wide range of frequencies. To this purpose, three types of components will be employed: resistors, capacitors and transistors.

The secondary goal is to achieve the best performance at the lowest cost. This is quantified by maximizing the following function:
\begin{equation}
  M = \frac{voltageGain \cdot bandwidth}{cost \cdot lowerCutoffFrequency}
\end{equation}
\begin{gather*}
  cost = cost\;of\;resistors + cost\;of\;capacitors + cost\;of\;transistors \\
  cost\;of\;resistors = 1 MU/k\Omega \\
  cost\;of\;capacitors = 1 MU/\mu F \\
  cost\;of\;transistors = 0.1 MU/transistor
\end{gather*}

In Section~\ref{sec:simulation}, the amplifier is presented and its behaviour is studied using a simulation. This is followed by a theoretical analysis of the circuit, in Section~\ref{sec:analysis}. Both are compared in the final section, Section ~\ref{sec:conclusion}. Here, $M$ is also presented, allowing for some considerations to be made about the circuit designed. The simulation was run by using {\bf Ngspice} and the theoretical calculations were performed with {\bf Octave}.
