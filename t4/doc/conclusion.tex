\section{Conclusion}
\label{sec:conclusion}

%parte da conversa do t3
%This laboratory assignment had as the main objective the construction and analysis of an AC/DC converter. The chosen design is present in Figure~\ref{acdc_ours}. The goal was to design a circuit capable of converting 230 V of AC with frequency of 50 Hz to 12 V of DC. The analysis of the circuit was achieved by performing a simulation, using {\bf Ngspice}, and a theoretical analysis, using {\bf Octave}. By analysing the results obtained, we conclude that we were able to achieve a good result, the output voltage is very near from 12V and has a small ripple.

The merit figure of our circuit is: \input{../sim/merit}.

\begin{table}[H]
  \centering
  \begin{tabular}{|c|c|}
    \hline
        \input{../sim/out}
        \hline
  \end{tabular}
  \caption{Voltage gain, lower cutoff frequency and bandwith for the simulation}
  \label{tab:res_sim}
\end{table}

%Some deviations between the simulation and theoretical results were expected, due to the different model used by {\bf Ngspice} for diodes. In the theoretical analysis, we used very simple models. On the other hand, {\bf Ngspice} uses a model of the diode with 15 parameters. Comparing graphs \ref{sim_res}, \ref{fig:acdcoc} and \ref{fig:vo-12} allows us to conclude that the output voltages have a similar behaviour. However, we can see that, in the simulation, the Envelope Detector output voltage oscillates around 14V while in the theoretical analysis it oscillates around 15V. 
%All in all, the objetives were met. With more time and a lenghtier study, better results, especially in terms of reducing the ripple, could be obtained, and other solutions implemented.
