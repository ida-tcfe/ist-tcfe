\section{Conclusion}
\label{sec:conclusion}

%parte da conversa do t3
%This laboratory assignment had as the main objective the construction and analysis of an AC/DC converter. The chosen design is present in Figure~\ref{acdc_ours}. The goal was to design a circuit capable of converting 230 V of AC with frequency of 50 Hz to 12 V of DC. The analysis of the circuit was achieved by performing a simulation, using {\bf Ngspice}, and a theoretical analysis, using {\bf Octave}. By analysing the results obtained, we conclude that we were able to achieve a good result, the output voltage is very near from 12V and has a small ripple.

The merit figure of our first circuit is: \input{../sim/merit}. For our second circuit, it's \input{../sim/merit_mb}.


\begin{table}[H]
  \centering
  \begin{tabular}{|c|c|c|c|c|}
    \hline
        $V$ & $V_t$ & $V_s$ & $|V_t-V_s|$ & $Error (\%)$ \\
        \hline
        \hline
        $V_{B1}$ (V) & 0.81181 & 1.529471 & 0.717661 & 46.92 \\ 
 \hline 
$V_{C1}$ (V) & 10.776937 & 11.19816 & 0.42122299999999946 & 3.76 \\ 
 \hline 
$V_{E1}$ (V) & 0.11181 & 0.8257281 & 0.7139181 & 86.46 \\ 
 \hline 
$V_{E2}$ (V) & 11.476937 & 11.96096 & 0.48402300000000054 & 4.05 \\ 
 \hline 
$I_{B1}$ (V) & 0.006222 & 4.350727e-05 & 0.00617849273 & 14201.06 \\ 
 \hline 
$I_{R1}$ (V) & 0.00048 & 0.0001163392 & 0.0003636608 & 312.59 \\ 
 \hline 
$I_{R2}$ (V) & 0.005742 & 7.283194e-05 & 0.005669168059999999 & 7783.90 \\ 
 \hline 
$I_{C1}$ (V) & 1.111876 & -0.00821377 & 1.12008977 & 13636.73 \\ 
 \hline 
$I_{E1}$ (V) & 1.118098 & 0.008257281 & 1.1098407190000001 & 13440.75 \\ 
 \hline 
$I_{E2}$ (V) & 0.174354 & -0.0390369 & 0.2133909 & 546.64 \\ 
 \hline 

        \hline
  \end{tabular}
  \caption{Theoretical values ($V_t$) and Simulation values ($V_s$) of the operating point analysis (table produced with {\bf Python})  - The absolute deviation and error presented here are rounded up to one significant digit, for ease of interpretation.}
  \label{error1_res}
\end{table}

%Some deviations between the simulation and theoretical results were expected, due to the different model used by {\bf Ngspice} for diodes. In the theoretical analysis, we used very simple models. On the other hand, {\bf Ngspice} uses a model of the diode with 15 parameters. Comparing graphs \ref{sim_res}, \ref{fig:acdcoc} and \ref{fig:vo-12} allows us to conclude that the output voltages have a similar behaviour. However, we can see that, in the simulation, the Envelope Detector output voltage oscillates around 14V while in the theoretical analysis it oscillates around 15V. 
%All in all, the objetives were met. With more time and a lenghtier study, better results, especially in terms of reducing the ripple, could be obtained, and other solutions implemented.
