\section{Simulation Analysis}
\label{sec:simulation}

\subsection{Operating Point Analysis}

\par After analysing the circuit we proceeded to the simulation using \bf{ngspice} in order to
compare the results with the theoretical predictions. The results are shown in Table~\ref{tab:op}

Table~\ref{tab:op} shows the simulated operating point results for the circuit
under analysis. Compared to the theoretical analysis results, one notices the
following differences: describe and explain the differences.

\begin{table}[h]
	\centering
	\begin{tabular}{|l|r|}
		\hline    
		{\bf Name} & {\bf Value [A or V]} \\ \hline
		n2 (V) & 5.070727e+00\\ \hline
n3 (V) & 4.825468e+00\\ \hline
n4 (V) & 4.304579e+00\\ \hline
n5 (V) & 4.860091e+00\\ \hline
n6 (V) & 8.720760e+00\\ \hline
n7 (V) & -2.93990e+00\\ \hline
n8 (V) & -1.95020e+00\\ \hline
Vc (V) & 7.799989e+00\\ \hline
r1 (A) & -2.40330e-04\\ \hline
r2 (A) & -2.51475e-04\\ \hline
r3 (A) & -1.11453e-05\\ \hline
r4 (A) & -1.20919e-03\\ \hline
r5 (A) & -1.25166e-03\\ \hline
r6 (A) & 9.688613e-04\\ \hline
r7 (A) & 9.688613e-04\\ \hline
Ib (A) & -2.51475e-04\\ \hline

	\end{tabular}
	\caption{Operating point. A variable preceded by @ is of type {\em current}
		and expressed in Ampere; other variables are of type {\it voltage} and expressed in
		Volt.}
	\label{tab:op}
\end{table}

\par As one can see there's a slight deviation in the values compared with previous calculations,
however this error is assumed to be caused by numerical propagated errors made when solving the linear 
system of equations. %There's also a certain random error...



%\subsection{Transient Analysis}

%Figure~\ref{fig:trans} shows the simulated transient analysis results for the
%circuit under analysis. Compared to the theoretical analysis results, one
%notices the following differences: describe and explain the differences.

%\begin{figure}[h] \centering
%\includegraphics[width=0.6\linewidth]{trans.pdf}
%\caption{Transient output voltage}
%\label{fig:trans}
%\end{figure}


