\section{Theoretical Analysis}
\label{sec:analysis}

In this section, the circuit depicted in Figure 'circuit' is analysed with two different methods, so as to determine how it behaves, theoretically, both in terms of current in each branch and potential difference between nodes.

\subsection{Mesh analysis method}

The first method considered is the Mesh analysis method, considering the four meshes present in the circuit. Each mesh is given a label ($\alpha$, $\beta$, $\gamma$ and $\delta$) and an arbitrary direction for the current to flow in, as the below figure shows.

%Boneco
%No boneco depois é preciso pôr as setinhas ou os +- que convencionam os sentidos da diferença de potencial
%Os sentidos que considerei estão no script do octave

By inspection:

\begin{equation}
  \begin{cases}
    I_{\alpha} = I_d \\
    I_{\beta} = I_b
  \end{cases}
\end{equation}

Applying KVL to the two remaining meshes:

\begin{equation}
  \begin{cases}
    V_a + R_4 (I_\gamma + I_\delta) + R_3 (I_\gamma - I_\beta) + R_1 I_\gamma = 0 \\
    V_c + R_7 I_\delta + R_6 I_\delta + R_4 (I_\delta + I_\gamma) = 0
  \end{cases}
\end{equation}

The conditional sources behave as:

\begin{equation}
  \begin{cases}
  I_b = k_b V_b \\
  V_c = k_c I_c = - k_c I_\delta
  \end{cases}
\end{equation}

Lastly, by Ohm's Law

\begin{equation}
  V_b = R_3 (I_\beta - I_\gamma)
\end{equation}

Manipulating these equations yields:

\begin{equation}
  \begin{bmatrix}
  1 & 0 & 0 & 0 \\
  0 & 1-k_b R_3 & k_b R_3 & 0 \\
  0 & -R_3 & R_1+R_3+R_4 & R_4 \\
  0 & 0 & R_4 & R_4+R_6+R_7-k_c
  \end{bmatrix}
  \begin{bmatrix}
  I_\alpha \\
  I_\beta \\
  I_\gamma \\
  I_\delta
  \end{bmatrix}
  =
  \begin{bmatrix}
  I_d \\
  0 \\
  -V_a \\
  0
  \end{bmatrix}
\end{equation}

Solving this system using Octave,

\begin{equation}
  \begin{cases}
    I_\alpha \approx 1.002\; mA \\
    I_\beta \approx -0.25148\; mA \\
    I_\gamma \approx -0.24033\; mA \\
    I_\delta \approx -0.96886\; mA
  \end{cases}
\end{equation}

Using Ohm's Law for each resistance, we get:

\begin{equation}
  \begin{cases}
    V_1 = -0.24526\; V \\
    V_2 = -0.52089\; V \\
    V_3 = -0.034623\; V \\
    V_4 = -4.8601\; V \\
    V_5 = -3.8607\; V \\
    V_6 = -1.9502\; V \\
    V_7 = -0.98970\; V
  \end{cases}
\end{equation}

We can also determine the current and potential difference imposed by the conditional sources:

\begin{equation}
  \begin{cases}
    I_b = -0.25148\; mA \\
    V_c = 7.8\; V
  \end{cases}
\end{equation}
    

\subsection{Node analysis method}

%Isabel

\lipsum[1-1]


